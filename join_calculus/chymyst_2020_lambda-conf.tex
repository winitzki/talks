%% LyX 2.3.3 created this file.  For more info, see http://www.lyx.org/.
%% Do not edit unless you really know what you are doing.
\documentclass[english]{beamer}
\usepackage[T1]{fontenc}
\usepackage[latin9]{inputenc}
\setcounter{secnumdepth}{3}
\setcounter{tocdepth}{3}
\usepackage{babel}
\usepackage{calc}
\usepackage{amstext}
\usepackage{graphicx}
\PassOptionsToPackage{normalem}{ulem}
\usepackage{ulem}
\ifx\hypersetup\undefined
  \AtBeginDocument{%
    \hypersetup{unicode=true,pdfusetitle,
 bookmarks=true,bookmarksnumbered=false,bookmarksopen=false,
 breaklinks=false,pdfborder={0 0 1},backref=false,colorlinks=true}
  }
\else
  \hypersetup{unicode=true,pdfusetitle,
 bookmarks=true,bookmarksnumbered=false,bookmarksopen=false,
 breaklinks=false,pdfborder={0 0 1},backref=false,colorlinks=true}
\fi

\makeatletter
%%%%%%%%%%%%%%%%%%%%%%%%%%%%%% Textclass specific LaTeX commands.
% this default might be overridden by plain title style
\newcommand\makebeamertitle{\frame{\maketitle}}%
% (ERT) argument for the TOC
\AtBeginDocument{%
  \let\origtableofcontents=\tableofcontents
  \def\tableofcontents{\@ifnextchar[{\origtableofcontents}{\gobbletableofcontents}}
  \def\gobbletableofcontents#1{\origtableofcontents}
}
\newenvironment{lyxcode}
  {\par\begin{list}{}{
    \setlength{\rightmargin}{\leftmargin}
    \setlength{\listparindent}{0pt}% needed for AMS classes
    \raggedright
    \setlength{\itemsep}{0pt}
    \setlength{\parsep}{0pt}
    \normalfont\ttfamily}%
   \def\{{\char`\{}
   \def\}{\char`\}}
   \def\textasciitilde{\char`\~}
   \item[]}
  {\end{list}}

%%%%%%%%%%%%%%%%%%%%%%%%%%%%%% User specified LaTeX commands.
\usetheme[secheader]{Boadilla}
\usecolortheme{seahorse}
\author{Sergei Winitzki}
\date{August 11, 2020}
\setbeamertemplate{headline}{} % disable headline at top
\setbeamertemplate{navigation symbols}{} % disable navigation bar at bottom
\title[Declarative concurrency in Scala]{Declarative Concurrent Programming with Join Calculus in Scala}
\institute[$\Lambda$-Conf 2020]{Lambda-Conf 2020 Global Edition}

\makeatother

\begin{document}
\frame{\titlepage}
\begin{frame}{Chemical Machine: a new hope}


\framesubtitle{...and some new hype}

In this talk:
\begin{itemize}
\item ``join calculus'' as a ``\textbf{Chemical Machine}'', without
academic jargon
\begin{itemize}
\item ``$\lambda$-calculus'' --- a small (but complete) programming
language
\item ``join calculus'' --- a small (but complete) language for concurrency
\begin{itemize}
\item adds 2 features (``async mailboxes'' and ``async functions'')
to $\lambda$-calculus
\item in the chemical metaphor: ``molecules'' and ``reactions''
\end{itemize}
\end{itemize}
\item \texttt{Chymyst} -- open-source implementation of Chemical Machine
(Scala)
\item examples of concurrent programs in \texttt{Chymyst}
\begin{itemize}
\item implement anything in 10-15 lines of code
\end{itemize}
\item comparisons with Actor Model, ``AWS Lambda'', and Petri nets
\item an extension for distributed programming: DCM
\end{itemize}
\emph{Not} in this talk: \sout{academic theory}
\begin{itemize}
\item \sout{Petri net theory, \mbox{$\pi$}-calculus, join calculus,
joinads, formal semantics}
\item DCM defined \sout{within some formalism for distributed programming}
\end{itemize}
\end{frame}

\begin{frame}{The chemical metaphor I. ``Abstract'' chemistry}

Real chemistry is asynchronous, concurrent, and parallel:
\[
\text{HCl}+\text{NaOH}\rightarrow\text{NaCl}+\text{H}_{2}\text{O}
\]

Want to run \emph{computations} similarly to how chemical reactions
run!

Begin by formulating the execution model of ``abstract chemistry'':
\begin{itemize}
\item Abstract ``molecules'' float around in a ``chemical reaction site''
\item Certain kinds of molecules may combine to start a ``reaction'':
\end{itemize}
~\\

\fbox{\begin{minipage}[c][1\totalheight][t]{0.5\columnwidth}%
\begin{center}
``abstract'' chemical laws:\\
\texttt{\textcolor{blue}{\footnotesize{}a + b ${\color{blue}\rightarrow}$
a}}\\
\texttt{\textcolor{blue}{\footnotesize{}a + c ${\color{blue}\rightarrow}$
$\textrm{�}$}}
\par\end{center}%
\end{minipage}}\hfill{}%
\begin{minipage}[c][1\totalheight][t]{0.3\columnwidth}%
\includegraphics[width=1\columnwidth]{cham1a}%
\end{minipage}\hfill{}

~\\

\begin{itemize}
\item Program code defines molecules \texttt{\textcolor{blue}{\scriptsize{}a}},
\texttt{\textcolor{blue}{\scriptsize{}b}}, \texttt{\textcolor{blue}{\scriptsize{}c}},
... and chemical reactions
\item At initial time, the code emits some molecules into the site
\item The runtime system runs reactions \emph{concurrently and in parallel}
\begin{itemize}
\item A chemical simulation engine is easy to implement in any language
\end{itemize}
\end{itemize}
\end{frame}

\begin{frame}{The chemical metaphor II. Chemical Machine in a nutshell}


\framesubtitle{From abstract chemistry to computation}

Translating the chemical metaphor into a model of computation:~\\
~

\fbox{\begin{minipage}[c][1\totalheight][t]{0.5\columnwidth}%
\begin{itemize}
\item Each molecule carries a \textbf{value} (``concurrent data'')
\item Each reaction computes new values from its input values
\item Some molecules with new values may be emitted back into the reaction
site
\end{itemize}
%
\end{minipage}}\hfill{}%
\begin{minipage}[c][1\totalheight][t]{0.45\columnwidth}%
\includegraphics[width=1\columnwidth]{cham2}

~\\
\texttt{\textcolor{blue}{\scriptsize{}site(}}{\scriptsize\par}

\texttt{\textcolor{blue}{\scriptsize{}~ go \{ case }}\texttt{\textbf{\textcolor{blue}{\scriptsize{}a}}}\texttt{\textcolor{blue}{\scriptsize{}(x)
+ }}\texttt{\textbf{\textcolor{blue}{\scriptsize{}b}}}\texttt{\textcolor{blue}{\scriptsize{}(y)
$\Rightarrow$}}{\scriptsize\par}

\texttt{\textcolor{blue}{\scriptsize{}~ ~val z = f(x, y); }}\texttt{\textbf{\textcolor{blue}{\scriptsize{}a}}}\texttt{\textcolor{blue}{\scriptsize{}(z)
\},}}{\scriptsize\par}

\texttt{\textcolor{blue}{\scriptsize{}~ go \{ case }}\texttt{\textbf{\textcolor{blue}{\scriptsize{}a}}}\texttt{\textcolor{blue}{\scriptsize{}(x)
+ }}\texttt{\textbf{\textcolor{blue}{\scriptsize{}c}}}\texttt{\textcolor{blue}{\scriptsize{}(\_)
$\Rightarrow$}}{\scriptsize\par}

\texttt{\textcolor{blue}{\scriptsize{}~ ~ ~ println(x) \}}}{\scriptsize\par}

\texttt{\textcolor{blue}{\scriptsize{})}}{\scriptsize\par}%
\end{minipage}\hfill{}\\
~\\
When a reaction starts: input molecules disappear, new values are
computed, output molecules are emitted

Reactions are \emph{functions} from input values to output values
\begin{itemize}
\item Need to learn how to ``think in molecules and reactions''
\end{itemize}
\end{frame}

\begin{frame}{Programming the Chemical Machine using \texttt{Chymyst}}


\framesubtitle{How I learned to forget semaphores and to love concurrency}

Molecule emitters are values of type \texttt{\textcolor{blue}{\footnotesize{}M{[}A{]}}}:
\begin{lyxcode}
\textcolor{blue}{\footnotesize{}val~count~=~m{[}Int{]};~val~inc~=~m{[}Unit{]}}{\footnotesize\par}
\end{lyxcode}
Molecules are emitted by calling the emitter's \texttt{\textcolor{blue}{\footnotesize{}apply()}}method:
\begin{lyxcode}
\textcolor{blue}{\footnotesize{}count(0);~inc();~inc();~inc();}{\footnotesize\par}
\end{lyxcode}
Reactions are values of type \texttt{\textcolor{blue}{\footnotesize{}Reaction}}:
\begin{lyxcode}
\textcolor{blue}{\footnotesize{}val~r:~Reaction~=~go~\{~case~count(x)~+~inc(\_)~\ensuremath{\Rightarrow}~count(x~+~1)~\}}{\footnotesize\par}
\end{lyxcode}
\begin{itemize}
\item A reaction may be any partial function with a single \texttt{\textcolor{blue}{\footnotesize{}case}}
clause
\item Emitters (e.g., \texttt{\textcolor{blue}{\footnotesize{}count}}, \texttt{\textcolor{blue}{\footnotesize{}inc}})
must be already defined
\end{itemize}
Reactions are ``activated'' as a group, using a ``reaction site'':
\begin{lyxcode}
\textcolor{blue}{\footnotesize{}site(r1,~r2,~r3)}{\footnotesize\par}
\end{lyxcode}
\begin{itemize}
\item Molecules may be emitted after activating their consuming reactions 
\item A site \emph{must} include all reactions that contend on an input
molecule
\end{itemize}
Programs declare molecules and reaction sites, then emit initial molecules
\begin{itemize}
\item The \texttt{Chymyst} runtime runs the simulation
\end{itemize}
\end{frame}

\begin{frame}{Example: throttling}

Throttle emitting molecules \texttt{\textcolor{blue}{\footnotesize{}s(x)}}
with minimum allowed delay of \texttt{\textcolor{blue}{\footnotesize{}delta}}
ms
\begin{lyxcode}
\textcolor{blue}{\footnotesize{}def~throttle{[}X{]}(s:~M{[}X{]},~delta:~Long):~M{[}X{]}~=~\{}{\footnotesize\par}

\textcolor{blue}{\footnotesize{}~val~r~=~m{[}X{]}}{\footnotesize\par}

\textcolor{blue}{\footnotesize{}~val~enable~=~m{[}Unit{]}~}\textcolor{gray}{\scriptsize{}//~Molecule~is~confined~to~the~local~scope.}{\scriptsize\par}

\textcolor{blue}{\footnotesize{}~site(}{\footnotesize\par}

\textcolor{blue}{\footnotesize{}~~go~\{~case~r(x)~+~enable(\_)~\ensuremath{\Rightarrow}}{\footnotesize\par}

\textcolor{blue}{\footnotesize{}~~~~~~~~s(x)}{\footnotesize\par}

\textcolor{blue}{\footnotesize{}~~~~~~~~Thread.sleep(delta)}{\footnotesize\par}

\textcolor{blue}{\footnotesize{}~~~~~~~~enable()}{\footnotesize\par}

\textcolor{blue}{\footnotesize{}~~~~~\}}{\footnotesize\par}

\textcolor{blue}{\footnotesize{}~)}{\footnotesize\par}

\textcolor{blue}{\footnotesize{}~enable()~}\textcolor{gray}{\scriptsize{}//~Enable~emitting~`s`~initially.}{\scriptsize\par}

\textcolor{blue}{\footnotesize{}~r~}\textcolor{gray}{\scriptsize{}//~Outside~scope~will~be~able~to~emit~`r`.}{\scriptsize\par}

\textcolor{blue}{\footnotesize{}\}}{\footnotesize\par}
\end{lyxcode}
\begin{itemize}
\item No threads/semaphores/locks, no mutable state
\item External code may emit \texttt{\textcolor{blue}{\footnotesize{}r(x)}}
at will, and \texttt{\textcolor{blue}{\footnotesize{}s(x)}} is then
throttled
\item External code may not emit \texttt{\textcolor{blue}{\footnotesize{}enable()}} 
\end{itemize}
\href{\%20https://github.com/softwaremill/akka-vs-scalaz/tree/master/core/src/main/scala/com/softwaremill/ratelimiter}{Implementations in Akka, in Monix, and ZIO}:
> 50 LOC each
\end{frame}

\begin{frame}{Example: cancellation}

Cancel emitting molecules \texttt{\textcolor{blue}{\footnotesize{}x(a)}}
after a molecule \texttt{\textcolor{blue}{\footnotesize{}cancel()}}
is emitted
\begin{lyxcode}
\textcolor{blue}{\footnotesize{}def~cancellation{[}A{]}(x:~M{[}A{]}):~(M{[}A{]},~M{[}Unit{]})~=~\{}{\footnotesize\par}

\textcolor{blue}{\footnotesize{}~~val~cx~=~m{[}A{]}}\textcolor{gray}{\scriptsize{}~//~Cancellable~version~of~`x`.}{\scriptsize\par}

\textcolor{blue}{\footnotesize{}~~val~cancel~=~m{[}Unit{]};~val~enable~=~m{[}Unit{]}}{\footnotesize\par}

\textcolor{blue}{\footnotesize{}~~site(}{\footnotesize\par}

\textcolor{blue}{\footnotesize{}~~~~go~\{~case~cx(a)~+~enable(\_)~\ensuremath{\Rightarrow}~x(a);~enable()~\},}{\footnotesize\par}

\textcolor{blue}{\footnotesize{}~~~~go~\{~case~enable(\_)~+~cancel(\_)~\ensuremath{\Rightarrow}~\}}{\footnotesize\par}

\textcolor{blue}{\footnotesize{}~~)}{\footnotesize\par}

\textcolor{blue}{\footnotesize{}~~enable()~}\textcolor{gray}{\scriptsize{}//~Enable~emitting~`x`~initially.}{\scriptsize\par}

\textcolor{blue}{\footnotesize{}~~(cx,~cancel)~}\textcolor{gray}{\scriptsize{}//~Outside~scope~will~be~able~to~emit~these.}{\scriptsize\par}

\textcolor{blue}{\footnotesize{}\}}{\footnotesize\par}
\end{lyxcode}
\begin{itemize}
\item No threads/semaphores/locks, no mutable state
\item External code may not emit \texttt{\textcolor{blue}{\footnotesize{}enable()}}
because of local scope
\end{itemize}
Users cannot implement this in Monix or ZIO (cancellation is provided)
\end{frame}

\begin{frame}{An additional feature: Blocking molecules}

A molecule emitter type \texttt{\textcolor{blue}{\footnotesize{}B{[}A,
R{]}}} receives a reply value of type \texttt{\textcolor{blue}{\footnotesize{}R}}:
\begin{lyxcode}
\textcolor{blue}{\scriptsize{}val~fetch~=~b{[}Unit,~Int{]}}\textcolor{gray}{\scriptsize{}~//~Blocking~emitter.}{\scriptsize\par}

\textcolor{blue}{\scriptsize{}val~result~=~m{[}Int{]}}{\scriptsize\par}

\textcolor{blue}{\scriptsize{}site(}{\scriptsize\par}

\textcolor{blue}{\scriptsize{}~~go~\{~case~fetch(\_,~reply)~+~res(b)~$\Rightarrow$~reply(b)~\}}{\scriptsize\par}

\textcolor{blue}{\scriptsize{})}\textcolor{gray}{\footnotesize{}~}\textcolor{gray}{\scriptsize{}//~`reply`~is~a~ReplyEmitter{[}Int{]}~defined~within~reaction~scope.}{\scriptsize\par}

\textcolor{blue}{\scriptsize{}result(123)}{\scriptsize\par}

\textcolor{blue}{\scriptsize{}val~x~=~fetch()}\textcolor{gray}{\footnotesize{}~}\textcolor{gray}{\scriptsize{}//~Blocking~call,~will~set~x~=~123.}{\scriptsize\par}
\end{lyxcode}
~

Blocking emitters are a convenience, do not increase expressive power
\end{frame}

\begin{frame}{Example: map/reduce}

A simple map/reduce implementation:
\begin{lyxcode}
\textcolor{blue}{\scriptsize{}val~c~=~m{[}A{]}}\textcolor{gray}{\scriptsize{}~//~Initial~values~have~type~`A`.}{\scriptsize\par}

\textcolor{blue}{\scriptsize{}val~d~=~m{[}(Int,~B){]}~}\textcolor{gray}{\scriptsize{}//~`B`~is~a~commutative~monoid.}{\scriptsize\par}

\textcolor{blue}{\scriptsize{}val~res~=~m{[}B{]}}\textcolor{gray}{\scriptsize{}~//~Final~result~of~type~`B`.}{\scriptsize\par}

\textcolor{blue}{\scriptsize{}val~fetch~=~b{[}Unit,~B{]}}\textcolor{gray}{\scriptsize{}~//~Blocking~emitter.}{\scriptsize\par}

\textcolor{blue}{\scriptsize{}site(}{\scriptsize\par}

\textcolor{gray}{\scriptsize{}~~//~``map''}{\scriptsize\par}

\textcolor{blue}{\scriptsize{}~~go~\{~case~c(x)~$\Rightarrow$~d((1,~long\_computation(x)))~\},~}{\scriptsize\par}

\textcolor{blue}{\scriptsize{}~}\textcolor{gray}{\scriptsize{}~//~``reduce''}{\scriptsize\par}

\textcolor{blue}{\scriptsize{}~~go~\{~case~d((n1,~b1))~+~d((n2,~b2))~$\Rightarrow$}{\scriptsize\par}

\textcolor{blue}{\scriptsize{}~~~val~(newN,~newB)~=~(n1~+~n2,~b1~|+|~b2)}{\scriptsize\par}

\textcolor{blue}{\scriptsize{}~~~if~(newN~==~total)~res(newB)~else~d((newN,~newB))~}{\scriptsize\par}

\textcolor{blue}{\scriptsize{}~~\},}{\scriptsize\par}

\textcolor{blue}{\scriptsize{}~~go~\{~case~fetch(\_,~reply)~+~res(b)~$\Rightarrow$~reply(b)~\}}{\scriptsize\par}

\textcolor{blue}{\scriptsize{})}{\scriptsize\par}

\textcolor{blue}{\scriptsize{}(1~to~100).foreach(x~$\Rightarrow$~c(x))}{\scriptsize\par}

\textcolor{blue}{\scriptsize{}fetch()}\textcolor{gray}{\footnotesize{}~}\textcolor{gray}{\scriptsize{}//~Blocking~call~will~return~the~final~result.}{\scriptsize\par}
\end{lyxcode}
~\\
Compare with the \href{https://stackoverflow.com/questions/17291851/mapreduce-implementation-with-akka}{Akka implementation here}
(100+ LOC)
\end{frame}

\begin{frame}{Example: parallel merge-sort}

Reactions can be recursive

Complete \texttt{Chymyst} code: \href{https://github.com/Chymyst/jc-talk-2017-examples/blob/master/src/test/scala/io/chymyst/talk_examples/MergeSortSpec.scala}{MergeSortSpec.scala}
\begin{lyxcode}
\textcolor{blue}{\scriptsize{}val~mergesort~=~m{[}(Array{[}T{]},~M{[}Array{[}T{]}{]}){]}}{\scriptsize\par}

\textcolor{blue}{\scriptsize{}site(}{\scriptsize\par}

\textcolor{blue}{\scriptsize{}~~go~\{~case~mergesort((arr,~}\textcolor{brown}{\scriptsize{}sortedResult}\textcolor{blue}{\scriptsize{}))~$\Rightarrow$}{\scriptsize\par}

\textcolor{blue}{\scriptsize{}~~~~if~(arr.length~<=~1)~sortedResult(arr)}{\scriptsize\par}

\textcolor{blue}{\scriptsize{}~~~~~~else~\{}{\scriptsize\par}

\textcolor{blue}{\scriptsize{}~~~~~~~~val~}\textcolor{brown}{\scriptsize{}sorted1}\textcolor{blue}{\scriptsize{}~=~m{[}Array{[}T{]}{]}}{\scriptsize\par}

\textcolor{blue}{\scriptsize{}~~~~~~~~val~}\textcolor{brown}{\scriptsize{}sorted2}\textcolor{blue}{\scriptsize{}~=~m{[}Array{[}T{]}{]}}{\scriptsize\par}

\textcolor{blue}{\scriptsize{}~~~~~~~~}\textcolor{brown}{\scriptsize{}site}\textcolor{blue}{\scriptsize{}(}\textsf{\textcolor{gray}{\footnotesize{}~//~Define~a~lower-level~reaction~site.}}{\footnotesize\par}

\textcolor{blue}{\scriptsize{}~~~~~~~~~~go~\{~case~}\textcolor{brown}{\scriptsize{}sorted1}\textcolor{blue}{\scriptsize{}(x)~+~}\textcolor{brown}{\scriptsize{}sorted2}\textcolor{blue}{\scriptsize{}(y)~$\Rightarrow$}{\scriptsize\par}

\textcolor{brown}{\scriptsize{}~~~~~~~~~~~~~sortedResult}\textcolor{blue}{\scriptsize{}(arrayMerge(x,y))}{\scriptsize\par}

\textcolor{blue}{\scriptsize{}~~~~~~~~~~~~~\}}{\scriptsize\par}

\textcolor{blue}{\scriptsize{}~~~~~~~~)}{\scriptsize\par}

\textcolor{blue}{\scriptsize{}~~~~~~~~val~(part1,~part2)~=~arr.splitAt(arr.length/2)}{\scriptsize\par}

\textcolor{blue}{\scriptsize{}~~~~~~~~}\textsf{\textcolor{gray}{\footnotesize{}//~Emit~lower-level~}}\textcolor{gray}{\footnotesize{}mergesort}\textsf{\textcolor{gray}{\footnotesize{}~molecules:}}{\footnotesize\par}

\textcolor{blue}{\scriptsize{}~~~~~~~~mergesort(part1,~}\textcolor{brown}{\scriptsize{}sorted1}\textcolor{blue}{\scriptsize{})~+~mergesort(part2,~}\textcolor{brown}{\scriptsize{}sorted2}\textcolor{blue}{\scriptsize{})}{\scriptsize\par}

\textcolor{blue}{\scriptsize{}~~~~\}}{\scriptsize\par}

\textcolor{blue}{\scriptsize{}\})}{\scriptsize\par}
\end{lyxcode}
\href{https://gist.github.com/stephenmcd/7edbcfb632c373eaf466}{Implementation in Akka}:
30 LOC for the same functionality
\end{frame}

\begin{frame}{Dining philosophers I. Declarative vs.~non-declarative code}


\framesubtitle{The paradigmatic example of concurrency, parallelism and resource
contention}

\href{https://en.wikipedia.org/wiki/Dining_philosophers_problem}{Five philosophers sit at a round table},
taking turns eating and thinking for random time intervals
\begin{center}
\includegraphics[height=4cm]{An_illustration_of_the_dining_philosophers_problem}
\par\end{center}

Problem: simulate the process, avoiding deadlock and starvation

Solutions in various programming languages: see \href{https://rosettacode.org/wiki/Dining_philosophers}{Rosetta Code}
\begin{itemize}
\item The Chemical Machine code is purely declarative
\end{itemize}
\end{frame}

\begin{frame}{Dining philosophers II. Implementation in \texttt{Chymyst}}


\framesubtitle{Five Dining Philosophers implemented in 15 lines of code}

Philosophers \texttt{\textcolor{blue}{\scriptsize{}1, 2, 3, 4, }}\textcolor{blue}{\scriptsize{}5}
and forks \texttt{\textcolor{blue}{\scriptsize{}f12, f23, f34, f45,
f51}}{\scriptsize\par}
\begin{lyxcode}
\textsf{\textcolor{gray}{\footnotesize{}//~...~definitions~of~emitters,~think(),~eat()~omitted~for~brevity}}{\footnotesize\par}

\textcolor{blue}{\scriptsize{}site~(}{\scriptsize\par}

\textcolor{blue}{\scriptsize{}~~go~\{~case~t1(\_)~$\Rightarrow$~think(1);~h1()~\},}{\scriptsize\par}

\textcolor{blue}{\scriptsize{}~~go~\{~case~t2(\_)~$\Rightarrow$~think(2);~h2()~\},}{\scriptsize\par}

\textcolor{blue}{\scriptsize{}~~go~\{~case~t3(\_)~$\Rightarrow$~think(3);~h3()~\},}{\scriptsize\par}

\textcolor{blue}{\scriptsize{}~~go~\{~case~t4(\_)~$\Rightarrow$~think(4);~h4()~\},}{\scriptsize\par}

\textcolor{blue}{\scriptsize{}~~go~\{~case~t5(\_)~$\Rightarrow$~think(5);~h5()~\},}{\scriptsize\par}

~ 

\textcolor{blue}{\scriptsize{}~~go~\{~case~h1(\_)~+~f12(\_)~+~f51(\_)~$\Rightarrow$~eat(1);~t1()~+~f12()~+~f51()~\},}{\scriptsize\par}

\textcolor{blue}{\scriptsize{}~~go~\{~case~h2(\_)~+~f23(\_)~+~f12(\_)~$\Rightarrow$~eat(2);~t2()~+~f23()~+~f12()~\},}{\scriptsize\par}

\textcolor{blue}{\scriptsize{}~~go~\{~case~h3(\_)~+~f34(\_)~+~f23(\_)~$\Rightarrow$~eat(3);~t3()~+~f34()~+~f23()~\},}{\scriptsize\par}

\textcolor{blue}{\scriptsize{}~~go~\{~case~h4(\_)~+~f45(\_)~+~f34(\_)~$\Rightarrow$~eat(4);~t4()~+~f45()~+~f34()~\},}{\scriptsize\par}

\textcolor{blue}{\scriptsize{}~~go~\{~case~h5(\_)~+~f51(\_)~+~f45(\_)~$\Rightarrow$~eat(5);~t5()~+~f51()~+~f45()~\}}{\scriptsize\par}

\textcolor{blue}{\scriptsize{})}{\scriptsize\par}

\textcolor{blue}{\scriptsize{}t1()~+~t2()~+~t3()~+~t4()~+~t5()}{\scriptsize\par}

\textcolor{blue}{\scriptsize{}f12()~+~f23()~+~f34()~+~f45()~+~f51()}~\\
\end{lyxcode}
Source code: \href{https://github.com/Chymyst/jc-talk-2017-examples/blob/master/src/main/scala/io/chymyst/talk_examples/DiningPhilosophers.scala}{DiningPhilosophers.scala}

For more examples, see the \href{https://github.com/Chymyst/chymyst-core}{code repository}
(first-of, barriers, rendezvous, critical sections, readers/writers,
Game of Life, elevators, etc.)
\end{frame}

\begin{frame}{Reasoning about code in the Chemical Machine paradigm}

Reasoning about concurrent data:
\begin{itemize}
\item Emit molecule with value $\approx$ lift data into the ``concurrent
world''
\item Define reaction $\approx$ lift a function into the ``concurrent
world''
\item Reaction site $\approx$ container for concurrent functions and data
\item Reaction consumes molecules $\approx$ function consumes input values
\item Reaction emits molecules $\approx$ function returns result values
\end{itemize}
Reasoning about code:
\begin{itemize}
\item What data do we need to handle concurrently? (Put it on molecules.)
\item What computations consume this data? (Define as reactions.)
\end{itemize}
Guarantees:
\begin{itemize}
\item Molecule emitters and reactions are immutable values in local scopes
\item Reaction sites are immutable once activated; can refactor to libraries
\item Multiple input molecules are consumed atomically by reactions
\end{itemize}
\end{frame}

\begin{frame}{Chemical Machine paradigm to become mainstream in 2033}

\vspace{-0.2cm}

\begin{itemize}
\item The gap from academic invention to industry adoption is 38.2 years
(infix math, continuations, $\lambda$-functions, OOP, CSP, map/reduce,
Actor Model, constraint programming, DAG dataflow, Hindley-Milner
types)
\end{itemize}
\begin{center}
\includegraphics[width=0.7\textwidth]{40-year-gap}
\par\end{center}
\begin{itemize}
\item The Chemical Machine paradigm was \href{http://citeseerx.ist.psu.edu/viewdoc/summary?doi=10.1.1.32.3078}{invented in 1995}
\end{itemize}
\end{frame}

\begin{frame}{Current features of \texttt{Chymyst}}

Experience with Chemical Machine programming is limited

Some features promise to be useful:
\begin{itemize}
\item Blocking molecules with timeouts and timeout back-signalling
\item Automatic pipelining of molecules (ordered mailboxes)
\item Thread pools, thread priority control, graceful shutdown
\item Facility to increase parallelism when using blocking code 
\item Errors in DSL are reported at compile-time or early run-time
\begin{itemize}
\item \texttt{Chymyst} uses Scala macros -- but only to inspect code
\item Static analysis enables optimization and error reporting
\end{itemize}
\item ``Static'' molecules with read-only access (similar to Akka ``agents'')
\begin{itemize}
\item Forgetting to emit a molecule is \#1 programmer error
\end{itemize}
\item Logging, debugging, unit-testing facilities
\end{itemize}
\end{frame}

\begin{frame}{Related frameworks: Petri nets}

Workflow management: an approach based on \href{https://en.wikipedia.org/wiki/Petri_net}{Petri nets}
\begin{itemize}
\item \href{https://github.com/ing-bank/baker}{ING Baker} -- a DSL for
workflow management, based on Petri nets
\item Process modeling and control (``elevator system'' etc.)
\item Business process management (BPM) systems
\end{itemize}
\texttt{Chymyst} implements a feature-rich version of Petri nets:
\begin{itemize}
\item Transitions admit arbitrary guard conditions and error recovery
\item Transitions carry values, reactions are values, can be nested
\item Nondeterministic, asynchronous, parallel execution
\end{itemize}
Any Petri net model is straightforwardly translated into a CM program
\end{frame}

\begin{frame}{Chemical Machine vs.~Amazon AWS Lambda}

How AWS$\lambda$ works:
\begin{itemize}
\item wait for an event that signals arrival of input data
\item run a computation whenever input data becomes available
\item the computation is automatically parallelized, data-driven
\item writing the output data will create a new event
\end{itemize}
Modify the AWS$\lambda$ execution model by adding new requirements:
\begin{itemize}
\item a Lambda should be able to wait for several \emph{unrelated} events
\item several Lambdas may contend \emph{atomically} on shared input events
\end{itemize}
With these new requirements, AWS$\lambda$ becomes ``AWS$\pi$''
-- a model of unrestricted concurrency
\begin{itemize}
\item (Implementation on AWS could be tricky)
\end{itemize}
\end{frame}

\begin{frame}{Chemical Machine vs.~the Actor model. I}

Modify the Actor execution model by adding new requirements:
\begin{itemize}
\item when messages arrive, actors are auto-created, maybe \emph{in parallel}
\item actors may wait atomically for messages in \emph{several} different
mailboxes
\end{itemize}
It follows from these requirements that... 
\begin{itemize}
\item Auto-created actor instances are \emph{stateless} and invisible to
user
\item User code defines \emph{mailboxes} and \emph{computations} that consume
messages
\item Repeated messages may be consumed in parallel
\item Messages are sent to mailboxes, not to specific actor instances:
\end{itemize}
\begin{minipage}[c][1\totalheight][t]{0.5\columnwidth}%
\begin{lyxcode}
\textcolor{blue}{\scriptsize{}//~Akka}{\scriptsize\par}

\textcolor{blue}{\scriptsize{}val~a:~ActorRef~=~...~receive(x)~$\Rightarrow$...}{\scriptsize\par}

\textcolor{blue}{\scriptsize{}val~b:~ActorRef~=~...~receive(y)~$\Rightarrow$...}{\scriptsize\par}

\textcolor{blue}{\scriptsize{}a~!~100}{\scriptsize\par}

\textcolor{blue}{\scriptsize{}b~!~1;~~~b~!~2;~~~b~!~3}{\scriptsize\par}
\end{lyxcode}
%
\end{minipage}\hfill{}%
\begin{minipage}[c][1\totalheight][t]{0.45\columnwidth}%
\begin{lyxcode}
\textcolor{blue}{\scriptsize{}//~Chymyst}{\scriptsize\par}

\textcolor{blue}{\scriptsize{}...~go~\{~case~a(x)~$\Rightarrow$~...~\}~}{\scriptsize\par}

\textcolor{blue}{\scriptsize{}...~go~\{~case~b(y)~+~c(z)~$\Rightarrow$~...~\}}{\scriptsize\par}

\textcolor{blue}{\scriptsize{}a(100)}{\scriptsize\par}

\textcolor{blue}{\scriptsize{}b(1);~~b(2);~~b(3);~c(\textquotedbl hello\textquotedbl );}{\scriptsize\par}
\end{lyxcode}
%
\end{minipage}\hfill{}
\begin{itemize}
\item All data resides on messages in mailboxes, is consumed automatically
\item Mailboxes and computations are \emph{values}, can be sent on messages
\end{itemize}
Any Actor program can be straightforwardly translated into CM
\end{frame}

\begin{frame}{Chemical Machine vs.~Actor model. II}

\begin{itemize}
\item \vspace{-0.3cm}
reaction $\approx$ function body for an (auto-started) actor
\item emitted molecule with value $\approx$ message with value, in a mailbox
\item molecule emitters $\approx$ mailbox references
\end{itemize}
Programming with actors: 
\begin{itemize}
\item user code creates and manages explicit actor instances
\item actors typically hold mutable state and/or mutate ``behavior''
\begin{itemize}
\item reasoning is about running processes \emph{and} the data sent on messages
\end{itemize}
\end{itemize}
Programming with the Chemical Machine:
\begin{itemize}
\item processes auto-start when the needed input molecules are available
\item many reactions may start at once, with automatic parallelism
\begin{itemize}
\item user code does not manipulate references to processes
\begin{itemize}
\item no state, no supervision, no lifecycle, no ``dead letters'', no
routers
\end{itemize}
\item reasoning is only about the \emph{data} \emph{currently} \emph{available}
on molecules
\begin{itemize}
\item no reasoning about running processes having internal state
\end{itemize}
\end{itemize}
\end{itemize}
\texttt{Chymyst} code is typically 2x -- 3x shorter than equivalent
Akka code
\end{frame}

\begin{frame}{Distributed Chemical Machine}


\framesubtitle{Run concurrent code on a cluster with no code changes}
\begin{itemize}
\item \vspace{-0.2cm}
Declare some molecules as ``distributed'', of type \texttt{\textcolor{blue}{\scriptsize{}DM{[}T{]}}} 
\item No other new language constructions are necessary!
\begin{itemize}
\item early prototype in progress, as extension of \texttt{Chymyst} 
\end{itemize}
\end{itemize}
Distributed map/reduce in 15 LOC:
\begin{lyxcode}
\textcolor{blue}{\scriptsize{}implicit~val~cluster~=~ClusterConfig(???)}{\scriptsize\par}

\textcolor{blue}{\scriptsize{}val~c~=~dm{[}Int{]}~;~val~d~=~dm{[}Int{]}}\textcolor{gray}{\scriptsize{}~//~distributed}{\scriptsize\par}

\textcolor{blue}{\scriptsize{}val~res~=~m{[}(Int,~List{[}Int{]}){]}~}\textcolor{gray}{\scriptsize{}//~local}{\scriptsize\par}

\textcolor{blue}{\scriptsize{}val~fetch~=~b{[}Unit,~List{[}Int{]}{]}}{\scriptsize\par}

\textcolor{blue}{\scriptsize{}site(}{\scriptsize\par}

\textcolor{blue}{\scriptsize{}~~go~\{~case~c(x)~$\Rightarrow$~d(x~{*}~2)~\},}\textcolor{gray}{\scriptsize{}~~//~``map''~on~cluster,}\textcolor{blue}{\scriptsize{}~}{\scriptsize\par}

\textcolor{blue}{\scriptsize{}~}\textcolor{gray}{\scriptsize{}//~``reduce''~on~the~driver~node~only.}{\scriptsize\par}

\textcolor{blue}{\scriptsize{}~~go~\{~case~res((n,~list))~+~d(x)~$\Rightarrow$~res((n-1,~s::list))~\},}{\scriptsize\par}

\textcolor{blue}{\scriptsize{}~}\textcolor{gray}{\scriptsize{}//~fetch~results}{\scriptsize\par}

\textcolor{blue}{\scriptsize{}~~go~\{~case~fetch(\_,~reply)~+~res((0,~list))~$\Rightarrow$~reply(list)~\}}{\scriptsize\par}

\textcolor{blue}{\scriptsize{})}{\scriptsize\par}

\textcolor{blue}{\scriptsize{}if~(isDriver)~\{}\textcolor{gray}{\scriptsize{}~//~`true`~only~on~the~driver~node.}{\scriptsize\par}

\textcolor{blue}{\scriptsize{}~~Seq(1,~2,~3).foreach(x~$\Rightarrow$~c(x))}{\scriptsize\par}

\textcolor{blue}{\scriptsize{}~~res((3,~Nil))~;~fetch()}\textcolor{gray}{\scriptsize{}~//~Returns~the~result.}{\scriptsize\par}

\textcolor{blue}{\scriptsize{}\}}{\scriptsize\par}
\end{lyxcode}
Comparison: \href{https://github.com/ltronky/MapReduce-akka}{Akka implementation of distributed map/reduce}
(400+ LOC)
\end{frame}

\begin{frame}{Distributed cache in 10 LOC}

\begin{itemize}
\item \vspace{-0.2cm}
Mutable \texttt{\textcolor{blue}{\scriptsize{}Map{[}String, String{]}}}
with operations: \texttt{\textcolor{blue}{\scriptsize{}put}}, \texttt{\textcolor{blue}{\scriptsize{}get}},
\texttt{\textcolor{blue}{\scriptsize{}delete}}{\scriptsize\par}
\end{itemize}
\begin{lyxcode}
\textcolor{blue}{\scriptsize{}implicit~val~cluster~=~ClusterConfig(???)}{\scriptsize\par}

\textcolor{blue}{\scriptsize{}val~data~=~dm{[}mutable.Map{[}String,~String{]}{]}}{\scriptsize\par}

\textcolor{blue}{\scriptsize{}val~put~=~dm{[}(String,~String){]}}{\scriptsize\par}

\textcolor{blue}{\scriptsize{}val~get~=~dm{[}(String,~M{[}Option{[}String{]}{]}{]}}{\scriptsize\par}

\textcolor{blue}{\scriptsize{}val~delete~=~dm{[}String{]}}{\scriptsize\par}

\textcolor{blue}{\scriptsize{}site(}{\scriptsize\par}

\textcolor{blue}{\scriptsize{}~go~\{~case~data(dict)~+~put((k,~v))~$\Rightarrow$~data(dict.updated(k,~v))~\},}{\scriptsize\par}

\textcolor{blue}{\scriptsize{}~go~\{~case~data(dict)~+~get((k,~r))~$\Rightarrow$~data(dict);~r(dict.get(k))~\},}{\scriptsize\par}

\textcolor{blue}{\scriptsize{}~go~\{~case~data(dict)~+~delete(k)~$\Rightarrow$~dict.remove(k);~data(dict)~\}}{\scriptsize\par}

\textcolor{blue}{\scriptsize{})}{\scriptsize\par}

\textcolor{blue}{\scriptsize{}if~(isDriver)~data(mutable.Map{[}String,~String{]}())}{\scriptsize\par}
\end{lyxcode}
\begin{itemize}
\item Comparison: \href{https://medium.com/@hussachai/creating-a-distributed-cache-in-100-lines-with-akka-5387bd7310fd}{Distributed cache in 100 lines of Akka}
\end{itemize}
\end{frame}

\begin{frame}{Distributed peer-to-peer chat in 15 LOC}

\begin{itemize}
\item \vspace{-0.2cm}
Register user names in chat room
\item Fetch list of users
\item Send and receive text messages
\end{itemize}
\begin{lyxcode}
\textcolor{blue}{\scriptsize{}implicit~val~cluster~=~ClusterConfig(???)}{\scriptsize\par}

\textcolor{blue}{\scriptsize{}val~users~=~dm{[}List{[}DM{[}String{]}{]}{]}}\textcolor{gray}{\scriptsize{}~//~List~of~users'~message~emitters.}{\scriptsize\par}

\textcolor{blue}{\scriptsize{}val~carrier~=~dm{[}DM{[}String{]}{]}}\textcolor{gray}{\scriptsize{}~//~Carries~this~node's~message~emitter.}{\scriptsize\par}

\textcolor{blue}{\scriptsize{}val~fetch~=~b{[}Unit,~List{[}DM{[}String{]}{]}{]}}{\scriptsize\par}

\textcolor{blue}{\scriptsize{}site(go~\{~case~users(es)~+~carrier(e)~$\Rightarrow$~users(e~::~es)~\}}{\scriptsize\par}

\textcolor{blue}{\scriptsize{},~go~\{~case~users(es)~+~fetch(\_,~r)~$\Rightarrow$~users(es);~r(es)~\}~)}{\scriptsize\par}

\textcolor{blue}{\scriptsize{}~~}{\scriptsize\par}

\textcolor{blue}{\scriptsize{}val~peerName~=~???}\textcolor{gray}{\scriptsize{}~//~Read~from~config~on~node.}{\scriptsize\par}

\textcolor{blue}{\scriptsize{}val~sender~=~new~DM{[}String{]}(peerName)~}\textcolor{gray}{\scriptsize{}//~Assign~unique~molecule~name.}{\scriptsize\par}

\textcolor{blue}{\scriptsize{}site(go~\{~case~sender(x)~\ensuremath{\Rightarrow}~println(s\textquotedbl Peer~\$peerName~reads~\$x\textquotedbl )\})}{\scriptsize\par}

\textcolor{blue}{\scriptsize{}~~~}{\scriptsize\par}

\textcolor{blue}{\scriptsize{}carrier(sender)}{\scriptsize\par}

\textcolor{blue}{\scriptsize{}if~(isDriver)~users(Nil)}{\scriptsize\par}

\textcolor{gray}{\scriptsize{}//~Fetch~list~of~users~and~send~a~message~to~Sergei~if~present.}{\scriptsize\par}

\textcolor{blue}{\scriptsize{}fetch()}{\scriptsize\par}

\textcolor{blue}{\scriptsize{}~~.find(\_.name~==~\textquotedbl Sergei\textquotedbl )}{\scriptsize\par}

\textcolor{blue}{\scriptsize{}~~.foreach(sender~$\Rightarrow$~sender(\textquotedbl hello\textquotedbl ))}{\scriptsize\par}
\end{lyxcode}
\begin{itemize}
\item Comparison: \href{https://github.com/typesafehub/activator-akka-clustering/tree/master/src/main/scala/chat}{Distributed chat in > 100 lines of Akka}
\end{itemize}
\end{frame}

\begin{frame}{Reasoning in the Distributed Chemical Machine}

Distributed computing is made declarative
\begin{itemize}
\item Determine which data needs to be distributed and/or concurrent
\item Determine which computations will need to consume that data
\item Emit initial molecules and let the DCM run
\end{itemize}
Pure peer-to-peer architecture:
\begin{itemize}
\item Distributed molecules may be consumed by \emph{any} DCM peer
\item All DCM peers operate in the same way (no master/worker)
\item All DCM peers need to define the same distributed reaction sites
\begin{itemize}
\item Non-DCM code may differ between peers
\item Code or configuration could designate DCM peer as a ``driver'' or
have different roles
\end{itemize}
\end{itemize}
\end{frame}

\begin{frame}{Chemical Machine: implementation details}

\begin{itemize}
\item Each reaction site has a scheduler thread and a worker thread pool
\item Each molecule is ``bound'' to a unique reaction site
\item Each emitted molecule is stored in a multi-set at its reaction site
\item Each emitted molecule triggers a search for possible reactions
\begin{itemize}
\item Reaction search proceeds concurrently for different reaction sites
\end{itemize}
\item Reactions are scheduled on the worker thread pool
\begin{itemize}
\item The thread pool can be configured per-reaction or per-site
\end{itemize}
\item Scala macros are used for static analysis and optimizations
\begin{itemize}
\item Automatically pipelined molecules
\item Simplify and analyze Boolean conditions
\end{itemize}
\item Error analysis is also performed at early run time
\begin{itemize}
\item Reaction site with errors remain inactive
\end{itemize}
\end{itemize}
\end{frame}

\begin{frame}{Distributed Chemical Machine: implementation details}

\begin{itemize}
\item Each distributed molecule (DM) is bound to a unique reaction site
\item Emitted DM data goes into the ZK instance
\item Each DCM peer listens to ZK messages and checks for its DMs
\begin{itemize}
\item Once a DM is found, its data is downloaded and deserialized
\end{itemize}
\item On a DCM peer, each DM is identified with a unique local RS
\begin{itemize}
\item Downloaded molecules are emitted into the local RS to run reactions
\item All DCM peers must run identical reaction code for DMs
\end{itemize}
\item Each DCM peer acquires a distributed lock on its DMs
\begin{itemize}
\item Lock is released once reaction scheduling is complete
\end{itemize}
\item If a node goes down or network fails, molecules will be \emph{unconsumed}
\begin{itemize}
\item Another DCM peer will pick up these molecules later
\end{itemize}
\end{itemize}
\end{frame}

\begin{frame}{Conclusions and outlook}

\begin{itemize}
\item Chemical Machine = declarative, purely functional concurrency
\begin{itemize}
\item Enough power to replace threads, semaphors, atomic vars, etc.
\item Similar to Actor Model, but easier to use and ``more purely functional''
\item Significantly shorter code, easier to reason about
\end{itemize}
\item An open-source Scala implementation: \texttt{\href{https://github.com/Chymyst/chymyst-core}{Chymyst}}
\begin{itemize}
\item Static DSL code analysis (with Scala macros)
\item Industry-strength features (thread priority control, pipelining, fault
tolerance, unit testing and debugging APIs)
\item Extensive documentation: \href{https://winitzki.gitbooks.io/concurrency-in-reactions-declarative-multicore-in/content/}{tutorial book}
\item Distributed Chemical Machine -- work in progress
\end{itemize}
\item Promising applications:
\begin{itemize}
\item Workflow management, BPM
\item Asynchronous GUIs
\item Distributed peer-to-peer systems
\end{itemize}
\end{itemize}
\end{frame}

\end{document}
