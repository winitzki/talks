%% LyX 2.0.6 created this file.  For more info, see http://www.lyx.org/.
%% Do not edit unless you really know what you are doing.
\documentclass[english,hyperref]{beamer}
\usepackage[T1]{fontenc}
\usepackage[latin9]{inputenc}
\setcounter{secnumdepth}{3}
\setcounter{tocdepth}{3}
\usepackage{babel}
\usepackage{amsmath}
\usepackage{amssymb}
\usepackage{graphicx}
\ifx\hypersetup\undefined
  \AtBeginDocument{%
    \hypersetup{unicode=true,pdfusetitle,
 bookmarks=true,bookmarksnumbered=true,bookmarksopen=false,
 breaklinks=false,pdfborder={0 0 1},backref=false,colorlinks=true,
 linkcolor=black,urlcolor=blue}
  }
\else
  \hypersetup{unicode=true,pdfusetitle,
 bookmarks=true,bookmarksnumbered=true,bookmarksopen=false,
 breaklinks=false,pdfborder={0 0 1},backref=false,colorlinks=true,
 linkcolor=black,urlcolor=blue}
\fi
\usepackage{breakurl}

\makeatletter

%%%%%%%%%%%%%%%%%%%%%%%%%%%%%% LyX specific LaTeX commands.
%% Because html converters don't know tabularnewline
\providecommand{\tabularnewline}{\\}

%%%%%%%%%%%%%%%%%%%%%%%%%%%%%% Textclass specific LaTeX commands.
 % this default might be overridden by plain title style
 \newcommand\makebeamertitle{\frame{\maketitle}}%
 \AtBeginDocument{
   \let\origtableofcontents=\tableofcontents
   \def\tableofcontents{\@ifnextchar[{\origtableofcontents}{\gobbletableofcontents}}
   \def\gobbletableofcontents#1{\origtableofcontents}
 }
 \long\def\lyxframe#1{\@lyxframe#1\@lyxframestop}%
 \def\@lyxframe{\@ifnextchar<{\@@lyxframe}{\@@lyxframe<*>}}%
 \def\@@lyxframe<#1>{\@ifnextchar[{\@@@lyxframe<#1>}{\@@@lyxframe<#1>[]}}
 \def\@@@lyxframe<#1>[{\@ifnextchar<{\@@@@@lyxframe<#1>[}{\@@@@lyxframe<#1>[<*>][}}
 \def\@@@@@lyxframe<#1>[#2]{\@ifnextchar[{\@@@@lyxframe<#1>[#2]}{\@@@@lyxframe<#1>[#2][]}}
 \long\def\@@@@lyxframe<#1>[#2][#3]#4\@lyxframestop#5\lyxframeend{%
   \frame<#1>[#2][#3]{\frametitle{#4}#5}}
 \def\lyxframeend{} % In case there is a superfluous frame end
 \newenvironment{lyxcode}
   {\par\begin{list}{}{
     \setlength{\rightmargin}{\leftmargin}
     \setlength{\listparindent}{0pt}% needed for AMS classes
     \raggedright
     \setlength{\itemsep}{0pt}
     \setlength{\parsep}{0pt}
     \normalfont\ttfamily}%
    \def\{{\char`\{}
    \def\}{\char`\}}
    \def\textasciitilde{\char`\~}
    \item[]}
   {\end{list}}

%%%%%%%%%%%%%%%%%%%%%%%%%%%%%% User specified LaTeX commands.
\usetheme[secheader]{Boadilla}
\usecolortheme{seahorse}
\title[Temporal Logic and FRP]{Temporal Logic and Functional Reactive Programming}
\author{Sergei Winitzki}
\date{April 25, 2014}
\institute[Versal Group Inc.]{Bay Area Categories and Types}

\makeatother

\begin{document}
\frame{\titlepage}


\lyxframeend{}\lyxframe{What is ``reactive programming''}

\begin{center}
\begin{tabular}{|c|c|}
\hline 
\textbf{Transformational} programs & \textbf{Reactive} programs\tabularnewline
\hline 
\hline 
example: \texttt{\textcolor{blue}{pdflatex frp\_talk.tex}} & example: any GUI program\tabularnewline
\hline 
start, run, then stop & keep running indefinitely\tabularnewline
\hline 
read some input, write some output & wait for signals, send messages \tabularnewline
\hline 
\textbf{execution:} sequential + some parallel & ``main run loop'' + concurrency\tabularnewline
\hline 
\textbf{difficulty:} algorithms & signal/response sequences\tabularnewline
\hline 
\textbf{specification:} classical logic & temporal logic? flowcharts?\tabularnewline
\hline 
\textbf{verification:} prove it correct  & model checking?\tabularnewline
\hline 
\textbf{synthesis:} extract code from proof & temporal synthesis? \tabularnewline
\hline 
\textbf{type theory:} intuitionistic logic & intuitionistic temporal logic\tabularnewline
\hline 
\end{tabular}
\par\end{center}


\lyxframeend{}


\lyxframeend{}\lyxframe{The uses of logic in computer science }
\begin{enumerate}
\item Logic as a \textbf{specification} \textbf{language} - enables automatic
verification 

\begin{itemize}
\item Automatic synthesis of programs from specifications?
\end{itemize}
\item (Intuitionistic) logic as \textbf{type theory} - guides language design

\begin{itemize}
\item Mathematicians have already minimized the set of axioms! 
\end{itemize}
\item \textbf{Logic programming} - use a decidable subset of logic 

\begin{itemize}
\item Very high-level language, but limited to its domain 
\end{itemize}
\item \textbf{Automated theorem proving} - derive a program as a proof artifact 

\begin{itemize}
\item Requires advanced type systems and proof heuristics
\end{itemize}
\end{enumerate}

\lyxframeend{}


\lyxframeend{}\lyxframe{Part 1: Introduction to temporal logic}
\begin{itemize}
\item Let's forget all philosophy, ``what is time'', ``what is true'',
modal logic...
\item We want to see logic as a down-to-earth, computationally useful tool
\item We begin with the computational view of classical Boolean logic
\end{itemize}

\lyxframeend{}


\lyxframeend{}\lyxframe{Boolean algebra: notation}
\begin{itemize}
\item Classical propositional (Boolean) logic: $T$, $F$, $a\vee b$, $a\wedge b$,
$\neg a$, $a\rightarrow b$
\item A notation better adapted to school-level algebra: $1$, $0$, $a+b$,
$ab$, $a'$
\item The only ``new rule'' is $1+1=1$
\item Define $a\rightarrow b=a'+b$
\item Some identities: 
\begin{align*}
0a=0,\quad1a=a, & \quad a+0=a,\quad a+1=1,\\
a+a=a,\quad aa=a, & \quad a+a'=1,\quad aa'=0,\\
\left(a+b\right)'=a'b', & \quad\left(ab\right)'=a'+b',\quad\left(a'\right)'=a\\
a\left(b+c\right)=ab+ac, & \quad\left(a+b\right)\left(a+c\right)=a+bc
\end{align*}

\item DNF = ``expand all brackets''. Some DNF simplification tricks: 
\begin{align*}
a+ab=a, & \quad a\left(a+b\right)=a,\\
\left(a\rightarrow b\right)\left(a'\rightarrow c\right) & =ab+a'c,\\
\left(a\rightarrow x\right)\left(b\rightarrow x'\right) & =\left(a'+x\right)\left(b'+x'\right)=x'a'+xb'
\end{align*}

\end{itemize}

\lyxframeend{}


\lyxframeend{}\lyxframe{Boolean algebra: example}

\textcolor{blue}{\emph{Of the three suspects $A$, $B$, $C$, only
one is guilty of a crime. }}

\textcolor{blue}{\emph{Suspect $A$ says: ``$B$ did it''. Suspect
$B$ says: ``$C$ is innocent.''}}

\textcolor{blue}{\emph{The guilty one is lying, the innocent ones
tell the truth.}}
\[
\phi=\left(ab'c'+a'bc'+a'b'c\right)\left(a'b+ab'\right)\left(b'c'+bc\right)
\]
\textbf{Simplify}: expand the brackets, omit $aa'$, $bb'$, $cc'$,
replace $aa=a$ etc.:
\[
\phi=ab'c'+0+0=ab'c'
\]


\textcolor{blue}{The guilty one is $A$.}


\lyxframeend{}


\lyxframeend{}\lyxframe{Synthesis of Boolean programs}
\begin{itemize}
\item Specification of a ``Boolean program'':
\end{itemize}
\textcolor{blue}{\emph{If the boss is in, I need to work unless the
telephone rings.}}

\textcolor{blue}{\emph{If the boss is not in, I go drink tea.}}
\begin{itemize}
\item $b=$``boss is in'', $r=$``telephone rings'', $w=$``I work'',
$w'=$``I drink tea''
\begin{align*}
\phi(b,r,w) & =\left(br'\rightarrow w\right)\left(b'\rightarrow w'\right)\\
 & =w'\left(br'\right)'+wb=w'\left(b'+r\right)+wb
\end{align*}

\item \textbf{Goal}: given any $b$ and $r$, compute $w$ such that $\phi(b,r,w)=1$. 
\item One solution is just $\phi(b,r,w=1)$:
\[
w=\phi(b,r,1)=0\left(b'+r\right)+1b=b
\]
\textcolor{blue}{``I work if and only if the boss is in''}
\item (Other solutions exist, e.g. $w=br'$)
\end{itemize}

\lyxframeend{}


\lyxframeend{}\lyxframe{Propositional linear-time temporal logic (LTL)}
\begin{itemize}
\item Reactive programs respond to an \emph{infinite} \emph{stream} of signals
\item So let's work with\emph{ infinite boolean sequences} (``linear time'')\\
\textbf{Boolean} operations:
\begin{align*}
a & =\left[a_{0},a_{1},a_{2},...\right];\quad b=\left[b_{0},b_{1},b_{2},...\right];\\
a+b & =\left[a_{0}+b_{0},a_{1}+b_{1},...\right];\; a'=\left[a_{0}^{\prime},a_{1}^{\prime},...\right];\; ab=\left[a_{0}b_{0},a_{1}b_{1},...\right]
\end{align*}
\textbf{Temporal} operations:
\begin{align*}
\mbox{(Next)}\quad\mathbf{N}a & =\left[a_{1},a_{2},...\right]\\
\mbox{(Sometimes)}\quad\mathbf{F}a & =\left[a_{0}+a_{1}+a_{2}+...,\ a_{1}+a_{2}+...,\ ...\right]\\
\mbox{(Always)}\quad\mathbf{G}a & =\left[a_{0}a_{1}a_{2}a_{3}...,\ a_{1}a_{2}a_{3}...,\ a_{2}a_{3}...,\ ...\right]
\end{align*}
Other notation (from modal logic):
\[
\mathbf{N}a\equiv\bigcirc a;\;\mathbf{F}a\equiv\lozenge a;\;\mathbf{G}a\equiv\square a
\]
\texttt{\textcolor{blue}{\scriptsize{ }}}{\scriptsize \par}
\end{itemize}

\lyxframeend{}


\lyxframeend{}\lyxframe{Temporal fixpoints and the $\mu$-calculus notation}
\begin{itemize}
\item LTL admits \emph{only} temporal functions defined\emph{ by} \emph{fixpoints}:
\begin{align*}
\mathbf{F}a & =\left[a_{0}+a_{1}+a_{2}+a_{3}+...,\ a_{1}+a_{2}+a_{3}+...,\ ...\right]\\
\mathbf{F}a & =a+\mathbf{N}(\mathbf{F}a)\\
\mathbf{G}a & =\left[a_{0}a_{1}a_{2}a_{3}...,\ a_{1}a_{2}a_{3}...,a_{2}a_{3}...,\ ...\right]\\
\mathbf{G}a & =a\mathbf{N}(\mathbf{G}a)
\end{align*}

\item Notation: $\boldsymbol{\mu}$ for the least fixpoint, $\boldsymbol{\nu}$
for the greatest fixpoint
\begin{align*}
\mathbf{F}a=\boldsymbol{\mu}x.\left(a+\mathbf{N}x\right); & \quad\mathbf{G}a=\boldsymbol{\nu}x.\left(a(\mathbf{N}x)\right)\\
\mbox{but}\quad\boldsymbol{\nu}x.\left(a+\mathbf{N}x\right)=1; & \quad\boldsymbol{\mu}x.\left(a(\mathbf{N}x)\right)=0
\end{align*}

\item The most general fixpoints involving \emph{only one} $\mathbf{N}$:
\begin{align*}
\mbox{(weak Until)}\quad p\mathbf{U}q & =\boldsymbol{\nu}x.\left(q+p(\mathbf{N}x)\right)\\
\mbox{(strict Until)}\quad p\dot{\mathbf{U}}q & =\boldsymbol{\mu}x.\left(q+p(\mathbf{N}x)\right)
\end{align*}

\end{itemize}

\lyxframeend{}


\lyxframeend{}\lyxframe{LTL: interpretation of ``Until''}
\begin{itemize}
\item Weak Until: $p\mathbf{U}q$ = ``$p$ holds from now on until $q$
first becomes true''
\[
p\mathbf{U}q=q+p\mathbf{N}\left(q+p\mathbf{N}\left(q+...\right)\right)
\]
Example:
\begin{align*}
a & =\left[1,0,0,0,1,0,...\right]\\
b & =\left[0,1,0,1,0,1,...\right]\\
a\mathbf{U}b & =\left[1,1,0,1,1,1,...\right]
\end{align*}

\item Strict Until: $p\dot{\mathbf{U}}q$ = ``$q$ \emph{must} become true,
and $p$ holds until then'' 
\item Dualities: $\left(\mathbf{F}a\right)'=\mathbf{G}(a')$; also $(p\dot{\mathbf{U}}q)^{\prime}=q'\mathbf{U}(p'q')$
\end{itemize}

\lyxframeend{}


\lyxframeend{}\lyxframe{LTL: temporal specification }

\textcolor{blue}{\emph{Whenever the boss comes by my office, I will
start working. }}

\textcolor{blue}{\emph{Once I start working, I will keep working until
the telephone rings.}}

\[
\mathbf{G}\left(\left(b\rightarrow\mathbf{F}w\right)\left(w\rightarrow w\mathbf{U}r\right)\right)=\mathbf{G}\left(\left(b'+\mathbf{F}w\right)\left(w'+w\mathbf{U}r\right)\right)
\]


\textcolor{blue}{\emph{Whenever the button is pressed, the dialog
will appear. }}

\textcolor{blue}{\emph{The dialog will disappear after 1 minute of
user inactivity.}}

\[
\mathbf{G}\left(\left(b\rightarrow\mathbf{F}d\right)\left(d\rightarrow\mathbf{F}t\right)\left(d\rightarrow d\mathbf{U}td'\right)\right)
\]

\begin{itemize}
\item The timer $t$ is an external event and is \emph{not specified} here
\item Difficult to say ``$x$ stays true until further notice'' 
\end{itemize}

\lyxframeend{}


\lyxframeend{}\lyxframe{LTL: disjunctive normal form}
\begin{itemize}
\item What would be the DNF in LTL? Let's just expand brackets:
\begin{align*}
\phi & =\mathbf{G}\left(\left(b'+\mathbf{F}w\right)\left(w'+w\mathbf{U}r\right)\right)=\left(b'+\mathbf{F}w\right)\left(w'+w\mathbf{U}r\right)\mathbf{N}\phi\\
 & =\left(b'+w+\mathbf{N}(\mathbf{F}w)\right)\left(w'+r+w\mathbf{N}(w\mathbf{U}r)\right)\mathbf{N}\phi\\
 & =\left(b'+w+\mathbf{N}(w+\mathbf{N}(\mathbf{F}w))\right)\left(w'+r+w\mathbf{N}(r+w\mathbf{N}(w\mathbf{U}r))\right)\mathbf{N}\left(...\right)\\
 & =\quad...\quad\mathbf{N}\left(...\quad...\mathbf{N}(...\quad...\mathbf{N}\left(...\right))\right)...
\end{align*}
We will \emph{never} \emph{finish} expanding those brackets!\medskip{}

\item But many subformulas under $\mathbf{N}(...)$ are the same:
\begin{align*}
\phi_{1} & =\mathbf{F}w;\quad\phi_{2}=w\mathbf{U}r;\\
\phi & =\left(b'+w+\mathbf{N}\phi_{1}\right)\left(w'+r+w\mathbf{N}\phi_{2}\right)\mathbf{N}\phi\\
 & =\left(rw+b'w'\right)\mathbf{N}\phi+w'\mathbf{N}(\phi\phi_{1})+w\mathbf{N}(\phi\phi_{2})
\end{align*}
 
\end{itemize}

\lyxframeend{}


\lyxframeend{}\lyxframe{LTL: disjunctive normal form}
\begin{itemize}
\item Let's expand and simplify $\phi\phi_{1}$ and $\phi\phi_{2}$: we
get simultaneous fixpoints
\begin{align*}
\phi & =\left(rw+b'w'\right)\mathbf{N}(\phi)+w'\mathbf{N}(\phi\phi_{1})+w\mathbf{N}(\phi\phi_{2});\\
\phi\phi_{1} & =rw\mathbf{N}(\phi)+\left(r+w'\right)\mathbf{N}(\phi\phi_{1})+w\mathbf{N}(\phi\phi_{2});\\
\phi\phi_{2} & =r\left(w+b'\right)\mathbf{N}(\phi)+r\mathbf{N}(\phi\phi_{1})+w\mathbf{N}(\phi\phi_{2}).
\end{align*}

\item The DNF for LTL is a \emph{graph!}
\end{itemize}
\begin{center}
\includegraphics[width=0.8\columnwidth]{tree-dnf}
\par\end{center}


\lyxframeend{}


\lyxframeend{}\lyxframe{The failure of LTL program synthesis}
\begin{itemize}
\item Goal: given $b$ and $r$, determine $w$ 
\item The DNF generates a \emph{nondeterministic} finite automaton (NFA)
for $w$
\item States of the automaton are $\phi$, $\phi\phi_{1}$, $\phi\phi_{2}$,
... (\emph{sets of fixpoints} of $\phi$)
\item Determinizing the automaton may exponentially increase its size

\begin{itemize}
\item Worst case: for $\phi$ with $n$ fixpoints, DFA will have $2^{2^{n}}$
states
\end{itemize}
\item Specifications will generally need to use many fixpoints. Example:\\
\textcolor{blue}{\emph{Whenever $b$ is pressed, send query $q$ and
show $w$ (``Waiting''). }}\\
\textcolor{blue}{\emph{Upon reply $r$, show $d$ (``Done''). Pressing
$c$ (``Cancel'') stops waiting. }}
\begin{align*}
\phi & =\mathbf{G}\,[\left(bw'\rightarrow b\mathbf{U}d'w\right)\left(w\rightarrow d'w\mathbf{U}\left(c+r\right)\right)\\
 & \quad\ \quad\left(cw\rightarrow c\mathbf{U}w'\right)\left(q\leftrightarrow bw'\right)\left(rw\rightarrow r\mathbf{U}dw'\right)].
\end{align*}


\begin{itemize}
\item LTL is not particularly convenient for modular specification
\item Synthesis is \emph{not practical} (I write and debug my automata by
hand...)
\end{itemize}
\end{itemize}

\lyxframeend{}


\lyxframeend{}\lyxframe{Part 2: Temporal logic as type theory}
\begin{itemize}
\item Logic gives a recipe for designing a \emph{minimal} programming language
(Curry-Howard isomorphism)
\item Typically, we use an \textbf{intuitionistic} version of the logic:

\begin{itemize}
\item No negation, no $\bot$; only $a+b$, $ab$, $a\rightarrow b$
\item No law of excluded middle
\item No truth tables, no ``simplification''
\item Usually, cannot derive proofs automatically
\end{itemize}
\item Axioms are \emph{predefined terms} needed in the language

\begin{itemize}
\item Example: $\left(a\rightarrow c\right)\rightarrow\left(b\rightarrow c\right)\rightarrow\left(a+b\rightarrow c\right)$
is the \textbf{case} operator
\end{itemize}
\item Proof rules are \emph{predefined constructions} needed in the language

\begin{itemize}
\item Example: modus ponens ($a$; $a\rightarrow b$ so $b$) is function
application
\end{itemize}
\item Natural deduction rules are typing rules for the language
\end{itemize}

\lyxframeend{}


\lyxframeend{}\lyxframe{Interpreting values typed by LTL}
\begin{itemize}
\item What does it mean to have a value $x$ of type, say, $\mathbf{G}(\alpha\rightarrow\alpha\mathbf{U}\beta)$?

\begin{itemize}
\item $x:\mathbf{N}\alpha$ means that $x:\alpha$ will be available \emph{only}
at the \emph{next} time tick \\
($x$ is a \textbf{deferred value} of type $\alpha$)
\item $x:\mathbf{F}\alpha$ means that $x:\alpha$ will be available at
\emph{some} future tick(s)\\
($x$ is an \textbf{event} of type $\alpha$)
\item $x:\mathbf{G\alpha}$ means that a (different) value $x:\alpha$ is
available at \emph{every} tick\\
($x$ is an \textbf{infinite stream} of type $\alpha$)
\item $x:\alpha\mathbf{U}\beta$ means a \textbf{finite stream} of $\alpha$
that may end with a $\beta$ 
\end{itemize}
\item Some \emph{temporal axioms} of intuitionistic LTL:
\begin{align*}
\mbox{(deferred apply)}\quad\mathbf{N}\left(\alpha\rightarrow\beta\right) & \rightarrow\left(\mathbf{N}\alpha\rightarrow\mathbf{N}\beta\right);\\
\mathbf{\mbox{(streamed apply)}\quad G}\left(\alpha\rightarrow\beta\right) & \rightarrow\left(\mathbf{G}\alpha\rightarrow\mathbf{G}\beta\right);\\
\mbox{(generate a stream)}\quad\mathbf{G}\left(\alpha\rightarrow\mathbf{N}\alpha\right) & \rightarrow\left(\alpha\rightarrow\mathbf{G}\alpha\right);\\
\mbox{(read infinite stream)}\quad\mathbf{G}\alpha & \rightarrow\alpha\mathbf{N}(\mathbf{G}\alpha)\\
\mbox{(read finite stream)}\quad\alpha\mathbf{U}\beta & \rightarrow\beta+\alpha\mathbf{N}(\alpha\mathbf{U}\beta)
\end{align*}

\end{itemize}

\lyxframeend{}


\lyxframeend{}\lyxframe{A small FRP language: Elm}
\begin{itemize}
\item Core \texttt{Elm}: polymorphic $\lambda$-calculus with \texttt{lift2},
\texttt{foldp}, \texttt{async} \end{itemize}
\begin{lyxcode}
lift2~:~$\left(\alpha\rightarrow\beta\rightarrow\gamma\right)\rightarrow\mathbf{G}\alpha\rightarrow\mathbf{G}\beta\rightarrow\mathbf{G}\gamma$

foldp~:~$\left(\alpha\rightarrow\beta\rightarrow\beta\right)\rightarrow\beta\rightarrow\mathbf{G}\alpha\rightarrow\mathbf{G}\beta$

async~:~$\mathbf{G}\alpha\rightarrow\mathbf{G}\alpha$\end{lyxcode}
\begin{itemize}
\item (\texttt{lift2} makes $\mathbf{G}$ an applicative functor)
\item \texttt{async} is a special \emph{scheduling} \emph{instruction}
\item Limitations:

\begin{itemize}
\item Cannot have a type $\mathbf{G}(\mathbf{G}\alpha)$, also no concept
of $\mathbf{N}$ or $\mathbf{F}$
\item Cannot construct temporal values by hand
\item This language is an \emph{incomplete} Curry-Howard image of LTL!
\item \textcolor{blue}{\emph{I work after the boss comes by and until the
phone rings}}: \\
$\quad$let after\_until w (b,r) = (w or b) and not r in \\
$\quad\quad$foldp after\_until false (boss, phone)
\end{itemize}
\end{itemize}

\lyxframeend{}


\lyxframeend{}\lyxframe{``Legacy'' FRP systems: \texttt{FrTime}, \texttt{Fran}, AFRP, etc.}
\begin{itemize}
\item Two functors: ``continuous behavior'' $\mathbf{G}\alpha$ and ``discrete
event'' $\mathbf{F}\alpha$
\item Time is conceptually continuous
\item Explicit $\mathbf{N}$, delay by time $\Delta t$, explicit values
of time
\item Many predefined combinators that do not follow from type theory: \\
\texttt{value-now}, \texttt{delay-by}, \texttt{integral}, ... (\texttt{FrTime})\\
\texttt{merge}, \texttt{switcher}, $\mathbf{G}\left(\mathbf{G}\alpha\right)$,
... (\texttt{Fran})
\item AFRP: temporal values are not available, only combinators!
\end{itemize}

\lyxframeend{}


\lyxframeend{}\lyxframe{A lower-level FRP language: \texttt{AdjS} }
\begin{itemize}
\item A lower-level type system: $\mathbf{N}\alpha$ (\texttt{next}), $\hat{\boldsymbol{\mu}}\alpha.\Sigma$
($\boldsymbol{\mu}$+next), $\boxdot\alpha$ (\texttt{stable})
\item Explicit one-step temporal fixpoints, for example $\mathbf{F}\tau=\hat{\boldsymbol{\mu}}\alpha.\tau+\alpha$
\[
\tau=\hat{\boldsymbol{\mu}}\alpha.\Sigma\cong\hat{\boldsymbol{\mu}}\alpha.\left[\frac{\mathbf{N}\tau}{\alpha}\right]\Sigma
\]

\item Term assignments, simplified (see Krishnaswamy's paper):
\begin{align*}
\frac{\Gamma\vdash e:\alpha}{\Gamma\vdash\mbox{next }e:\mathbf{N}\alpha}\;_{\mathbf{N}I}\quad\quad\frac{\Gamma\vdash f:\mathbf{N}\alpha\quad\Gamma,x:\alpha\vdash e:\beta}{\mbox{let next }x=f\mbox{ in }e:\beta}\;_{\mathbf{N}E}\\
\frac{\Gamma\vdash e:\left[\mathbf{N}(\hat{\boldsymbol{\mu}}\alpha.\Sigma)/\alpha\right]\Sigma}{\Gamma\vdash\mbox{into }e:\hat{\boldsymbol{\mu}}\alpha.\Sigma}\;_{\hat{\boldsymbol{\mu}}I}\quad\quad\frac{\Gamma\vdash e:\hat{\boldsymbol{\mu}}\alpha.\Sigma}{\Gamma\vdash\mbox{out }e:\left[\mathbf{N}(\hat{\boldsymbol{\mu}}\alpha.\Sigma)/\alpha\right]\Sigma}\;_{\hat{\boldsymbol{\mu}}E}\\
\frac{\Gamma\vdash e:\alpha}{\Gamma\vdash\mbox{stable }e:\boxdot\alpha}\;_{\boxdot I}\quad\quad\frac{\Gamma\vdash f:\boxdot\alpha\quad\Gamma,x:\alpha\vdash e:\beta}{\Gamma\vdash\mbox{let stable }x=f\mbox{ in }e:\beta}\;_{\boxdot E}
\end{align*}

\end{itemize}

\lyxframeend{}


\lyxframeend{}\lyxframe{Dreams of an ideal FRP language }
\begin{itemize}
\item Requirements for a broadly usable FRP language: 

\begin{itemize}
\item ``stable'' and ``temporal'' types distinguished statically
\item seamless conversion from \texttt{int} to $\mathbf{G}$(\texttt{int})
and back, for ``stable'' types
\item construct values of type $\mathbf{F}\alpha$ by hand: from asynchronous
scheduling
\item construct values of type $\mathbf{F}\alpha$ from external sources
(environment)
\item tick-free operation: $\mathbf{N}$ is not needed, use $\mathbf{F}$
instead
\item the runloop (UI thread / background threads) needs to be represented 
\end{itemize}
\item I guess we are still in the research phase here...
\end{itemize}

\lyxframeend{}


\lyxframeend{}\lyxframe{Conclusions and outlook}
\begin{itemize}
\item LTL is not a good fit as a specification language for reactive programs
\item LTL synthesis from specification is theoretical, not practical
\item FRP tries to make specification closer to implementation
\item There are some languages that implement FRP in various \emph{ad hoc}
ways
\item The ideal is not (yet) reached
\end{itemize}

\lyxframeend{}


\lyxframeend{}\lyxframe{Abstract}

In my day job, most bugs come from imperatively implemented reactive
programs. Temporal Logic and FRP are declarative approaches that promise
to solve my problems. I will briefly review the motivations behind
and the connections between temporal logic and FRP. I propose a rather
\textquotedbl{}pedestrian\textquotedbl{} approach to propositional
linear-time temporal logic (LTL), showing how to perform calculations
in LTL and how to synthesize programs from LTL formulas. I intend
to explain why LTL largely failed to solve the synthesis problem,
and how FRP tries to cope.\medskip{}


FRP can be formulated as a $\lambda$-calculus with types given by
the propositional intuitionistic LTL. I will discuss the limitations
of this approach, and outline the features of FRP that are required
by typical application programming scenarios. \medskip{}


My talk will be largely self-contained and should be understandable
to anyone familiar with Curry-Howard and functional programming.


\lyxframeend{}


\lyxframeend{}\lyxframe{Suggested reading }

E. Czaplicki, S. Chong. \href{http://people.seas.harvard.edu/~chong/pubs/pldi13-elm.pdf}{Asynchronous FRP for GUIs}.
(2013) 

E. Czaplicki. \href{http://www.seas.harvard.edu/sites/default/files/files/archived/Czaplicki.pdf}{Concurrent FRP for functional GUI}
(2012). 

N. R. Krishnaswamy. \href{https://www.mpi-sws.org/~neelk/simple-frp.pdfHigher-order functional reactive programming without spacetime leaks}{https://www.mpi-sws.org/$\sim$neelk/simple-frp.pdfHigher-order functional reactive programming without spacetime leaks}(2013).

M. F. Dam. Lectures on temporal logic. Slides: \href{http://www.csc.kth.se/~mfd/Courses/Temporal_logic/lecture1.pdf}{Syntax and semantics of LTL},
\href{http://www.csc.kth.se/~mfd/Courses/Temporal_logic/lecture2.pdf}{A Hilbert-style proof system for LTL} 

E. Bainomugisha, et al. \href{ftp://progftp.vub.ac.be/tech_report/2012/vub-soft-tr-12-13.pdf}{A survey of reactive programming}
(2013).

W. Jeltsch. \href{http://www.ioc.ee/~wolfgang/research/plpv-2013-paper.pdf}{Temporal logic with Until, Functional Reactive Programming with processes, and concrete process categories.}
(2013).

A. Jeffrey. \href{http://ect.bell-labs.com/who/ajeffrey/papers/plpv12.pdf}{LTL types FRP.}
(2012).

D. Marchignoli. \href{http://phd.di.unipi.it/Theses/PhDthesis_Marchignoli.pdf}{Natural deduction systems for temporal logic.}
(2002). -- See Chapter 2 for a natural deduction system for modal
and temporal logics. 


\lyxframeend{}
\end{document}
